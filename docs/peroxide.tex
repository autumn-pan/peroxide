\documentclass{article}
\usepackage{amsthm}
\usepackage{amsmath}
\usepackage{amssymb}
\usepackage[top=3cm, bottom=2.5cm, left=2.5cm, right=2.5cm]{geometry}

\newtheorem{definition}{Definition}
\newtheorem{lemma}{Lemma}[theorem]
\newtheorem{theorem}{Theorem}[section]
\newtheorem{corollary}{Corollary}[theorem]
\title{Head-Position Reachability Analysis: an Accelerated Model of Deciding Turing Machines}
\author{Autumn Pan}
\date{January 2026}
\begin{document}
	\maketitle
	\tableofcontents
	\newpage
	\section{Introduction}

	\subsection{Background}
	Many holdouts in S(6) may be solved by Backwards Reasoning (BR), that is, starting from the halting state and proving it unreachable has historically led to many holdouts being proven nonhalting. Despite these successes, deciders have relied primarily on contradiction, and have therefore been unable to decide certain families of Turing machines.

	Traditional Backwards Reasoning relies on constructing a tree of the possible configurations of a turing machine that will lead it to halt, essentially simulating the machine backwards. For some configurations $C_0$, there will exist no other configuration $C_1$ such that applying the machines transition function $\mu$ from $C_1$ will yield $C_0$ in exactly one step. Under traditional BR, this configuration would be deemed unreachable. If all leaves of this configuration tree are unreachable, then the machine is nonhalting. While this has seen success in deciding a number of Turing machines, the configuration tree often grows exponentially fast, making BR not only incredibly computationally expensive, but impractical for the majority of Turing machines. 

	\subsection{Head-Position Reachability}
	In this paper, we propose a new method of proving that certain configurations are unreachable: Head-Position Reachability (HPR). In HPR, we analyze the head's position relative to the tape segment, and determine whether or not it is possible for the head to be on that side of the tape segment. This enables us to determine the reachability of certain configurations that may otherwise fragment exponentially. We formalize reachability and Halting Configuration Trees, and introduce the notions of constructability, traversability, and critical direction, allowing Backwards Reasoning to systemically decide previously undecidable holdouts.
	\subsection{Applications}
	While it is not possible to make Backwards Analysis capable of deciding all Turing machines, HPR may allow for automated deciders that can decide new families of Turing machines, advancing our understanding of the halting behavior of small Turing machines. Moreover, since HPR is fundamentally a technique for determining the reachability of a halting configuration, this framework may be used in parallel with other emergent techniques in Backwards Reasoning to decide a broader range of turing machines. In particular, we also propose methods of showing unreachability in configuration trees by induction as opposed to just contradiction.

	\section{Halting Signatures and Reachability}
	The most fundamental idea in Backwards Reasoning is the halting signature. To describe how a Turing machine works in reverse, we must first define a way to focus on only the relevant portion of that turing machine.
	\begin{definition}[Halting Signature]
		We define a Halting signature $H^n$ of some machine $M$ as the tuple
		\begin{align*}
			(I, h, q, \tau, n)
		\end{align*}
		where
		\begin{itemize}
			\item $I$ is some finite interval $[a,b]$ within $\mathbb{Z}$
			\item $h \in I$ is the position of the Head
			\item $q$ is the machine's state
			\item $n$ is the number of steps it will take for $q$ to be the halting state
			\item $\tau:I\rightarrow\{0,1\}$ s.t for some cell $i\in I$, $\tau(i)$ defines the symbol that occupies $i$
		\end{itemize}
		We assume that $M$ provides a set of all states $Q$ with one and only one halt state $z\in q$, a tape alphabet ${0,1}$, and there exists some transition function $\mu$. Some halting signature of degree $n$ $H^n$ must halt in exactly $n$ steps, otherwise it is not a valid halting signature. We choose $a$ and $b$ such that the smallest and largest value $h$ will equal before halting are $a$ and $b$ respectively.
	\end{definition}
	Because the head will visit but not exceed both extremes of $I$, only cells in $I$ are relevant to whether or not that configuration will halt. If any halting configuration is reached within the runtime of a Turing machine, then it is halting. We will use this notion to define Halting Signature trees to fully capture the backwards behavior of a Turing Machine.

	Our goal with Halting Signatures is to use them to describe every possible configuration that halts under the rules of a Turing machine $M$ and show that all such configurations are unreachable. If we show that it is impossible for $M$ to reach a configuration that would eventually lead it to halt, then it must be nonhalting. But first, we must define what reachability is.
		\begin{definition}{Reachability}
		For some turing machine $M$, a halting signature $H^n$ is considered reachable if and only if there exists some $H^m$ and some $p \in \mathbb{N}$ where 
		\begin{align*}
		\mu^p(H^m)=H^n
		\end{align*}
		and $H^m$ is the starting configuration of $M$.
	\end{definition}
	This means that all unreachable halting signatures $H^n$ of a Turing machine $M$ will never appear in the runtime of $M$. The ultimate goal of Backwards Reasoning is to show that the halt state $H^0$ is unreachable, and therefore that $M$ is nonhalting.

	In order to analyze the reachability of halting signatures through a halting signature tree, we must first define the relationship between halting signatures. 
	\begin{definition}[Children]
		We say a halting signature $H^{n+1}$ is a child of $H^n$ if and only if applying the transition function $\mu$ to $H^{n+1}$ yields a halting signature that contains $H^n$, and we call the collection of all such $H^{n+1}$ the children of $H^n$. 
		Likewise, we say that a halting signature $H^m$ is a descendant of $H^n$ if and only if there exists some positive integer $a$ such that $\mu^a(H^m)$ contains $H^n$
	\end{definition}
	In a halting Turing machine, the starting configuration must be a descendant of the halt state $H^0$. To prove that a Turing machine $M$ is nonhalting, we assume for the purpose of contradiction that $M$ is halting, and show that the starting configuration cannot be a descendant of $H^0$.
	To do this, we consider every single configuration that halts under the rules of $M$ in the form of a Halting Signature Tree. A Halting Signature Tree is necessary for determining the unreachability of the halt state because it captures every possible way that $M$ can reach its halt state. If we can show that all Halting Signatures in this tree are unreachable, then we have shown that the starting configuration is not a descendant of $H^0$, and therefore that $M$ is nonhalting. It is especially important to use a Halting Signature Tree
	\begin{definition}{Halting Signature Tree}
		For some turing machine $M$ with exactly one halt state, let its halting signature tree (HST) $\Tau(M)$ be a directed tree with root $H^0$, and whose nodes are halting signatures $H^n$, and where there exists an edge
		\begin{align*}
			H^{n+1}\rightarrow H^n
		\end{align*}
		if and only if $H^{n+1}$ is a child of $H^n$
	\end{definition}
	\subsection{Reachability of Halting Signature Trees}
	It is impossible to prove every Halting Signature in a Halting Signature Tree unreachable. However, we can prove the reachability of halting signatures exhaustively by considering the reachability of their children.
	\begin{lemma} For all halting configurations $H^n$ of some Turing machine $M$, $H^n$ is reachable if and only if at least one child of $H^n$ is also reachable.
		\begin{proof}
			Assume that $H^n$ is reachable given $M$ has starting configuration $H_0$. By definition, the starting configuration of $M$ is a descendant of $H^n$. That is, there exists some integer $p$ such that
			\begin{align*}
				\mu^p(H_0) \text{ contains } H^n
			\end{align*}
			This means that there exists some halting signature $\mu^{p-1}(H_0) \text{ that contains } H^{n+1}$. Since applying the transition function $\mu$ to $\mu^{p-1}(H_0)$ contains $H^n$, we know that by definition, $H^{n+1}$ is a child of $H^n$. Therefore, if $H^n$ is reachable, at least one of its children $H^{n+1}$ is also reachable.
		\end{proof}
	\end{lemma}
	This means that given any subtree of the Halting Signature Tree rooted at $H^0$, if we can show that all leaves of this subtree are unreachable, then $H^0$ is also unreachable. Therefore, we can prove the unreachability of $H^0$ and by extension, the nonhalting of $M$ by showing that all leaves of a subtree of its Halting Signature Tree are unreachable. This is the fundamental idea behind Backwards Analysis. However, the Halting Signature Tree often splits exponentially, making it computationally impractical to solve using traditional methods. In the next section, we will describe Head-Position Reachability, which will allow us to prove the unreachability of certain halting signatures without having to consider every single child.

	\section{Head-Position Reachability}
	\subsection{Extremal Halting Signatures}
	For all halting signatures $H^n$ with indexed over $(a,b)\subset \mathbb{Z}$, we can split the head's position into three intervals:
	\begin{itemize}
		\item $(-\infty, a]$
		\item $(a, b)$
		\item $[b, \infty)$
	\end{itemize} 
	For all $H^n$ with $h\in(-\infty, a]\bigcup[b, \infty)$, we say that $H^n$ is an \emph{extremal halting signature}.

	\begin{definition}{Critical Direction}
		given an extre
	\end{definition}
\end{document}